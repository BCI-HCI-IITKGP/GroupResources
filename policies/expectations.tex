\documentclass[12pt,twoside]{article}
\usepackage[letterpaper, textwidth=6.5in, textheight=9in]{geometry}
\usepackage{amsmath}
\date{2015-2016 academic year}
\newcommand{\ve}[1]{\ensuremath{\mathbf{#1}}}
\usepackage{hyperref}
\title{Slaybaugh's Research Group \\ Nuclear Engineering, UCB}
\begin{document}
%-----------------------------------------------
\maketitle


\begin{center}
\section*{Expectations for Graduate Students}
\end{center}

\subsection*{Research}

\textit{Students are expected to pursue graduate study with passion and independent thought.} The pursuit of an advanced degree in engineering requires enthusiasm, persistence, and hard work. The goal is for PhD students to complete their degrees within five years and MS students within two years. To meet this goal, it is expected that students who study with Prof.\ Slaybaugh
%
\begin{itemize}
\item Will work at least \textbf{20 productive hours} per week conducting research while taking a full load of graduate courses.
\item Will work at least \textbf{40 productive hours} per week conducting research while not completing courses.
\item May take up to 3 weeks of vacation (15 weekdays), including Winter and Spring Breaks; they must notify Prof.\ Slaybaugh when they plan to take vacation.
\item PhD students are expected to publish at least three high-quality, peer-reviewed manuscripts (journal articles and/or conference papers).
\item MS students are expected to publish at least one high-quality, peer-reviewed manuscript.
\end{itemize}

%-----------------------------------------------
\subsection*{Coursework}

Students are expected to 
%
\begin{itemize}
\item Remain in good standing with the graduate school and the student's home department, including maintaining at least a 3.0 GPA.
\item Students are expected to consult with Dr.\ Slaybaugh each semester about their course plans to complete their degree requirements.
\item It is desirable that PhD students complete coursework after three years of full-time enrollment. 
\end{itemize}

%-----------------------------------------------%-----------------------------------------------
\subsection*{Interactions}

Students are expected to meet with Prof.\ Slaybaugh on a regular basis (typically weekly or bi-weekly) to discuss research progress, course work, and graduate student life in general. Students are also expected to attend all research group meetings. To facilitate efficient interaction, student should: 
%
\begin{itemize}
\item Host notes for one-on-one meetings, code projects, and \LaTeX \ documents on GitHub, Bitbucket, or equivalent.
\item Include \underline{deadlines} in the subject lines of emails to Prof.\ Slaybaugh when she needs to do something by a specific date.
\item Check Prof.\ Slaybaugh's availability via Google calendar (found here:\\ \href{http://bconnected.berkeley.edu/}{http://bconnected.berkeley.edu/}. If you go there you can enter my name or email in the ``find other calendars" location on the left) and send her a calendar invite to schedule a meeting.
\item Place/access any \textit{student-specific} files that should not be version controlled in \textit{your} Dropbox folder Prof.\ Slaybaugh will share with you. 
\item Place/access \textit{project-based} files that should not be version controlled in the \textit{project} Dropbox folder Prof.\ Slaybaugh will share with you.
\end{itemize}

%-----------------------------------------------%-----------------------------------------------
\subsection*{Extraneous}

\textit{Students are expected to be effective communicators.} This will be accomplished through formal and informal oral presentations and by writing manuscripts involving their research. Students will share their research findings with their advisor and the rest of the research group on a regular basis, including group meetings.  

Effective communication also means seeking help when one encounters difficulties! Running into snags is a normal, functional part of doing research. However, if after a reasonable effort you can not find a solution to a problem that you face, please do not continue in isolation hoping that you can work the problem through. Advisors and other students are a tremendous resource, and you are expected to seek them out if you need help.

\textit{Students are expected to be informed members of the scientific community.} This includes successfully completing course work, attending seminars and external scientific meetings (with my approval), and staying current in publications relevant to your research topic. Some journals of interest to our research group include: 
%
\begin{itemize}
\item Nuclear Science and Engineering, Annals of Nuclear Energy, Nuclear Engieering and Design, Nuclear Technology
\item Journal of Computational Physics
\item SIAM 
\end{itemize}

\textit{Students are expected to be collaborative members of the scientific community.} Students are expected to be supportive and helpful to others working at Berkeley. In addition, students should seek to collaborate with colleagues and others as appropriate.

\textit{Students are expected to be proactive members of the scientific community.} This includes proactively seeking additional or new directions for their research that enhances the quality and/or significance of the overall program and proactively seeking supplemental fellowships (such as DOE, CSGF, Rickover, NSF, and Hertz Fellowships, etc.) as well as other award opportunities.

\textit{Students are expected to be positive contributors to the UCB community by exhibiting exemplary character.} We want to facilitate a positive environment for our academic pursuits, and to accomplish this we would like to promote students being helpful, respectful, courteous, honest, and trustworthy. 

Students will abide by all line items outlined in the American Nuclear Society (ANS) Code of Ethics (\href{http://www.new.ans.org/about/coe/}{http://www.new.ans.org/about/coe/}). 

%-----------------------------------------------
%------------------------------------------------------------------------------------
\begin{center}
\subsection*{Useful Skills}
\end{center}

\begin{itemize}
\item the shell and *nix environment
\item Python/iPython and another language such as C++ or Fortran
\item GitHub (Slaybaugh's username is rachelslaybaugh)
\item \LaTeX
\item Use a reference manager such as Mendeley
\end{itemize}

Berkeley's chapter of the Hacker Within is a great place to develop and refine these skills.

%-----------------------------------------------
%------------------------------------------------------------------------------------
\begin{center}
\section*{Expectations for Professor Slaybaugh}
\end{center}

Students will receive training and experiences that will prepare them for successful careers in industry, government, or academia.

Students can expect a supportive environment that rewards creativity, passion, collaboration, and hard work.

Students can expect and ask for discussions on performance, research plans for the year ahead, opportunities for publications and presentations at scientific meetings, and career interests.

Students can expect their advisor to make the best possible effort to provide continued funding (up to the 3 or 5 year limit) based on research productivity, seniority, and available grant funds.

Students can expect assistance in establishing a network of scientific contacts and mentors.



\end{document}
\documentclass[12pt,twoside]{article}
\usepackage[letterpaper, textwidth=6.5in, textheight=9in]{geometry}
\usepackage{amsmath}
\date{2015-2016 academic year}
\newcommand{\ve}[1]{\ensuremath{\mathbf{#1}}}
\usepackage{hyperref}
\title{Slaybaugh's Research Group \\ Nuclear Engineering, UCB}
\begin{document}
%-----------------------------------------------
\maketitle


\begin{center}
\section*{Expectations for Graduate Students}
\end{center}

\subsection*{Research}

\textit{Students are expected to pursue graduate study with passion and independent thought.} The pursuit of an advanced degree in engineering requires enthusiasm, persistence, and hard work. The goal is for PhD students to complete their degrees within five years and MS students within two years. To meet this goal, it is expected that students who study with Prof.\ Slaybaugh
%
\begin{itemize}
\item Will work at least \textbf{20 productive hours} per week conducting research while taking a full load of graduate courses.
\item Will work at least \textbf{40 productive hours} per week conducting research while not completing courses.
\item May take up to 3 weeks of vacation (15 weekdays), including Winter and Spring Breaks; they must notify Prof.\ Slaybaugh when they plan to take vacation.
\item PhD students are expected to publish at least three high-quality, peer-reviewed manuscripts (journal articles and/or conference papers).
\item MS students are expected to publish at least one high-quality, peer-reviewed manuscript.
\end{itemize}

%-----------------------------------------------
\subsection*{Coursework}

Students are expected to 
%
\begin{itemize}
\item Remain in good standing with the graduate school and the student's home department, including maintaining at least a 3.0 GPA.
\item Students are expected to consult with Dr.\ Slaybaugh each semester about their course plans to complete their degree requirements.
\item It is desirable that PhD students complete coursework after three years of full-time enrollment. 
\end{itemize}

%-----------------------------------------------%-----------------------------------------------
\subsection*{Interactions}

Students are expected to meet with Prof.\ Slaybaugh on a regular basis (typically weekly or bi-weekly) to discuss research progress, course work, and graduate student life in general. Students are also expected to attend all research group meetings. To facilitate efficient interaction, student should: 
%
\begin{itemize}
\item Host notes for one-on-one meetings, code projects, and \LaTeX \ documents on GitHub, Bitbucket, or equivalent.
\item Include \underline{deadlines} in the subject lines of emails to Prof.\ Slaybaugh when she needs to do something by a specific date.
\item Check Prof.\ Slaybaugh's availability via Google calendar (found here:\\ \href{http://bconnected.berkeley.edu/}{http://bconnected.berkeley.edu/}. If you go there you can enter my name or email in the ``find other calendars" location on the left) and send her a calendar invite to schedule a meeting.
\item Place/access any \textit{student-specific} files that should not be version controlled in \textit{your} Dropbox folder Prof.\ Slaybaugh will share with you. 
\item Place/access \textit{project-based} files that should not be version controlled in the \textit{project} Dropbox folder Prof.\ Slaybaugh will share with you.
\end{itemize}

%-----------------------------------------------%-----------------------------------------------
\subsection*{Extraneous}

\textit{Students are expected to be effective communicators.} This will be accomplished through formal and informal oral presentations and by writing manuscripts involving their research. Students will share their research findings with their advisor and the rest of the research group on a regular basis, including group meetings.  

Effective communication also means seeking help when one encounters difficulties! Running into snags is a normal, functional part of doing research. However, if after a reasonable effort you can not find a solution to a problem that you face, please do not continue in isolation hoping that you can work the problem through. Advisors and other students are a tremendous resource, and you are expected to seek them out if you need help.

\textit{Students are expected to be informed members of the scientific community.} This includes successfully completing course work, attending seminars and external scientific meetings (with my approval), and staying current in publications relevant to your research topic. Some journals of interest to our research group include: 
%
\begin{itemize}
\item Nuclear Science and Engineering, Annals of Nuclear Energy, Nuclear Engieering and Design, Nuclear Technology
\item Journal of Computational Physics
\item SIAM 
\end{itemize}

\textit{Students are expected to be collaborative members of the scientific community.} Students are expected to be supportive and helpful to others working at Berkeley. In addition, students should seek to collaborate with colleagues and others as appropriate.

\textit{Students are expected to be proactive members of the scientific community.} This includes proactively seeking additional or new directions for their research that enhances the quality and/or significance of the overall program and proactively seeking supplemental fellowships (such as DOE, CSGF, Rickover, NSF, and Hertz Fellowships, etc.) as well as other award opportunities.

\textit{Students are expected to be positive contributors to the UCB community by exhibiting exemplary character.} We want to facilitate a positive environment for our academic pursuits, and to accomplish this we would like to promote students being helpful, respectful, courteous, honest, and trustworthy. 

Students will abide by all line items outlined in the American Nuclear Society (ANS) Code of Ethics (\href{http://www.new.ans.org/about/coe/}{http://www.new.ans.org/about/coe/}). 

%-----------------------------------------------
%------------------------------------------------------------------------------------
\begin{center}
\subsection*{Useful Skills}
\end{center}

\begin{itemize}
\item the shell and *nix environment
\item Python/iPython and another language such as C++ or Fortran
\item GitHub (Slaybaugh's username is rachelslaybaugh)
\item \LaTeX
\item Use a reference manager such as Mendeley
\end{itemize}

Berkeley's chapter of the Hacker Within is a great place to develop and refine these skills.

%-----------------------------------------------
%------------------------------------------------------------------------------------
\begin{center}
\section*{Expectations for Professor Slaybaugh}
\end{center}

Students will receive training and experiences that will prepare them for successful careers in industry, government, or academia.

Students can expect a supportive environment that rewards creativity, passion, collaboration, and hard work.

Students can expect and ask for discussions on performance, research plans for the year ahead, opportunities for publications and presentations at scientific meetings, and career interests.

Students can expect their advisor to make the best possible effort to provide continued funding (up to the 3 or 5 year limit) based on research productivity, seniority, and available grant funds.

Students can expect assistance in establishing a network of scientific contacts and mentors.



\end{document}
