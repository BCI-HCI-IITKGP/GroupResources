%--------------------------------------------------------------------
% Undegraduate advising course requirements 

% formatting
\documentclass[12pt]{article}
\usepackage[top=1in, bottom=1in, left=1in, right=1in]{geometry}

\usepackage{setspace}
\onehalfspacing

\setlength{\parindent}{0mm} \setlength{\parskip}{1em}


% packages
\usepackage{amssymb}
%% The amsthm package provides extended theorem environments
\usepackage{amsthm}
\usepackage{epsfig}
\usepackage{times}
\renewcommand{\ttdefault}{cmtt}
\usepackage{amsmath}
\usepackage{graphicx} % for graphics files

% Draw figures yourself
\usepackage{tikz} 

% The float package HAS to load before hyperref
\usepackage{float} % for psuedocode formatting
\usepackage{xspace}

% from Denovo methods manual
\usepackage{mathrsfs}
\usepackage[mathcal]{euscript}
\usepackage{color}
\usepackage{array}

\usepackage[pdftex]{hyperref}

\newcommand{\nth}{n\ensuremath{^{\text{th}}} }
\newcommand{\ve}[1]{\ensuremath{\mathbf{#1}}}
\newcommand{\macro}{\ensuremath{\Sigma}}
\newcommand{\vOmega}{\ensuremath{\hat{\Omega}}}

\newcommand{\cc}[1]{\ensuremath{\overline{#1}}}
\newcommand{\ccm}[1]{\ensuremath{\overline{\mathbf{#1}}}}


%--------------------------------------------------------------------
%--------------------------------------------------------------------
\begin{document}
\begin{center}
{\bf Undegraduate Advising Course Requirements \\
As of October, 2014}
\end{center}

\setlength{\unitlength}{1in}
\begin{picture}(6,.1) 
\put(0,0) {\line(1,0){6.25}}         
\end{picture}

\section*{Things to send me}


\section*{Requirements}
Engineering Degree (http://engineering.berkeley.edu/student-services/degree-requirements): there are System level, University level, COE level, and Department level requirements.
%
\begin{itemize}
\item System Level:
  \begin{itemize}
  \item \textit{Entry level writing} (http://writing.berkeley.edu/): Either pass AWPE or take college writing R1A.
  \item \textit{American history and institutions} (http://registrar.berkeley.edu/Registrar/Default.aspx\\?PageID=ahi.html): There are a large number of ways to satisfy this requirement.
  \end{itemize}
%
\item University Level:
  \begin{itemize}
  \item \textit{American cultures} (http://americancultures.berkeley.edu/): Many courses satisfy this requirement. 
  \end{itemize}  
%
\item College Level:
  \begin{itemize}
  \item \textit{Reading and composition} (http://ls-breadth.berkeley.edu/): Many courses satisfy this. Must be taken for a letter grade and completed by the end of sophomore year. 
  \item \textit{Humanities and social sciences} (http://engineering.berkeley.edu/student-services/\\degree-requirements/humanities-and-social-sciences): You must take at least six \\Berkeley courses that satisfy the College of Engineering’s Humanities and Social \\Sciences (H/SS) breadth requirement. Foreign language courses are among those that satisfy the H/SS requirement.
  \item \textit{In addition}, these criteria must be met:
    \begin{itemize}
    \item You need to maintain a minimum overall grade-point average of 2.00 (C average) and a minimum grade-point average of 2.00 in upper-division technical coursework required of your major.
    \item Your final 30 units must be completed in residence in the College of Engineering on the Berkeley campus, in two consecutive semesters.
    \item All technical courses (math, science and engineering), required of the major or not, must be taken for a letter grade, unless they are only offered pass/no pass.
    \end{itemize}
  \end{itemize}  
%
\item Department Level: A semester-by-semester schedule is spelled out here \\ http://engineering.berkeley.edu/academics/undergraduate-guide/academic-departments-\\programs/nuclear-engineering
   
\end{itemize}



\end{document}
